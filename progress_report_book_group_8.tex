\documentclass[12pt]{book}
\usepackage[inner=1.25in, outer=1in, top=1in, bottom=1in]{geometry}
\title{Automated Attendance System}
\author{Govind M J}
\usepackage{graphicx}
\usepackage[document]{ragged2e}
\usepackage{titlesec}
\usepackage{hyperref}

\titleformat{\chapter}[display]
{\normalfont\bfseries}{}{0pt}{\Huge}


\setlength{\parindent}{0pt}
\begin{document}
	\pagestyle{plain}
    \pagenumbering{roman}
    \setlength\columnsep{20pt}
    \setlength{\columnseprule}{1pt}
	\centering
    \LARGE
    \thispagestyle{empty}
    \textbf{Progress Report Book} \vfill
    \large
    \textbf{On} \vfill
    \huge
    \textbf{\Large Automated Attendance Management System\\ using\\ Facial Recognition}
    \Large \\[15pt]
    \textbf{19-202-0611 Mini Project} \vfill
    \vfill
    \large
    \textbf{By} \\ [6pt]
    \Large
   	27 - A S Swan \\
    42 - Govind M J \\
    51 - Joel John Varghese \\
    55 - Lakshmi Manoj \vfill
    
    \Large
    (B.Tech. Computer Science and Engineering - Batch A) \vfill
    \textbf{Under the Guidance of} \\[6pt]
    Dr. Sheena Mathew \vfill
    

    \vfill
    
    \includegraphics[width=0.20\textwidth]{./logo.png}\\[0.5cm]
    \textbf{School of Engineering\\ Cochin University of Science and Technology} \\
    [10pt]
    2023 \\
    
    \chapter*{Preface}
    \normalsize
    \paragraph{}
    
    \begin{flushleft}
    	This progress report book provides a comprehensive overview of the development and implementation of the attendance management system. The goal of this project is to create a user-friendly system for tracking student attendance, enabling teachers and administrators to easily view attendance records and generate reports. The system is web-based, allowing users to access it from any device with an internet connection. \\[6pt]
    	
    	The project team consists of four individual. Our team is committed in delivering a high-quality system that meets the needs of attendance management. \\[6pt]
    	
    	This progress report book will provide an overview of the project's objectives, scope, methodology, and deliverables. We will also discuss the challenges encountered during the development process, and the strategies implemented to overcome them. In addition, we will provide a detailed description of the system's features, including the user interface, data management, security, and reporting capabilities. \\[6pt]
    	
    	We hope that this progress report book will serve as a valuable resource, providing insight into the development process and the system's functionality.
    \end{flushleft}
    
    
    
    \tableofcontents


    \chapter{Week 1}
    \justifying
    \large
    08/03/2023
    
    
    \pagenumbering{arabic}
    \setcounter{page}{1}
    	
    \paragraph{}
    \section{Framed the abstract}
    In this week, we have framed the abstract of the project.  
    \section{Basic working}
    Develop a real-time system which utilizes
    the knowledge and concepts revolving around facial
    recognition using Artificial Intelligence to mark students’
    attendance. A model will be trained in order to recognize a
    batch of students during its training phase. In the real time
    testing environment, it will be able to recognize these
    students uniquely by their facial features, which will be linked
    to their personal student information in a database upon
    which the attendance status will be reflected. \\
    
    Teachers will have access to a mobile phone application
    which can scan the QR associated to a camera or use the one
    present in their very mobile system to take the snapshot of
    the classroom/course when they desire to take the
    attendance. The image will be captured and faces of students
    will be recognized by the model upon which attendance will
    be updated in the database. Teachers can later track
    attendance and export them as .csv or .xls files for future
    reference.
    
    \chapter{Week 2}
    \justifying
    \large
    15/03/2023
    
    \paragraph{}
    \section{Proposed the Modules}
    The different modules included in the projects were proposed.
    
	\begin{itemize}
		
		\item Face Detection Module: This module detects the faces of individuals in a given image or video stream. It identifies the location of faces within an image and is an essential part of face recognition systems.
		
		\item Face Recognition Module: This module recognizes the identity of individuals by comparing the detected face to a database of known faces. It uses machine learning algorithms to identify the person by comparing features of the face such as the distance between the eyes, the shape of the nose, and the contours of the face.
		
		\item Database Management Module: This module stores and manages the data of registered users, such as their name, ID number, and face template.
		
		\item Attendance Management Module: This module records and tracks attendance for each user. It logs the time and date of each attendance and stores it in a database.
		
		\item User Interface Module: This module provides a graphical user interface that allows users to interact with the system. Users can register their faces, view their attendance records, and perform other actions through the interface.
		
	\end{itemize}

	\section{Basic human recognition done}
	In this week, we have tried basic human recognition using OpenCV and an open source trained model \textbf{haarcascade\_frontalface\_default.xml} which worked successfully. \\[12pt]
	\normalsize
	\textbf{Basic human recognition Source:} \\
	\url{https://github.com/gobmj/Attendance.git}
	\\[6pt]
	\textbf{Trained data set Source:} \\ \url{https://github.com/opencv/opencv/blob/master/data/haarcascades/haarcascade\_frontalface\_default.xml}
   
    
    \chapter{Week 3}
    \justifying
    \large
    23/03/2023
    
    \paragraph{}
    \section{Detection of presence of face(s)}
        The presence of face(s) is done primarily with the help of OpenCV, CMake, dlib and face\_recognition libraries available in support with Python.  
    \section{Identification of person(s)}
    Identification of person(s) corresponding to the face(s) whose presence has been detected, utilizes the encodings created by face\_recognition library corresponding to each face in the input image data and compares it to the encodings generated during real-time image capture.
    

	\chapter{Week 4}
	\justifying
	\large
	29/03/2023
	
	\paragraph{}
	\section{Basic interface and environment setup}
	We have set up the basic interface for the basic utilities and an environment setup for the utilities.
	\section{Debugging and optimization of code}
	We are currently working on debugging and optimizing the code to ensure smooth performance and eliminate any errors. This involves thorough testing and analysis of the code to identify and resolve any issues that may arise.
	\section{Database and Management}
	As we have completed the initial setup and debugging phase, we are now moving forward to the database and attendance management aspect of the project. This involves designing and implementing a robust database system to store and manage attendance data efficiently.
	
\end{document}

%Add Next weeks here!!!
